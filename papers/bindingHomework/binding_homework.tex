\title{Names, Scopes, and Bindings exercises} \author{
        Martin Kozeny\\
        CSCI 4501: Programming Language Structure\\
        Spring 2011
        University of New Orleans
}
\date{\today}




\documentclass[5pt]{article}
\usepackage{graphicx}
\usepackage{amssymb}
\usepackage{amsmath}
\usepackage{qtree}
\usepackage{multicol}
%\usepackage{chemarrow}
\usepackage[utf8]{inputenc}


\setlength{\hoffset}{-2.3cm} 
\setlength{\voffset}{-3cm}
\setlength{\textheight}{24.0cm} 
\setlength{\textwidth}{16cm}


\begin{document}


\maketitle

\setcounter{section}{2}
\section{Names, Scopes, and Bindings exercises}


\newcommand{\Subsection}[1]{\Subsection{#1} \setcounter{figure}{3}}

 

\subsection{}

\paragraph{Question}

Indicate the binding time (e.g., when the language is designed, when the
program is linked, when the program begins execution, etc.) for each of the
following decisions in your favorite programming language and implementation.
Explain any answers you think are open to interpretation

\begin{itemize}
\item The number of built-in functions (math, type queries, etc.)
\item The variable declaration that corresponds to a particular variable reference (use)
\item The maximum length allowed for a constant (literal) character string
\item The referencing environment for a subroutine that is passed as a parameter
\item The address of a particular library routine
\item The total amount of space occupied by program code and data

\end{itemize}

\paragraph{Answer}
\setcounter{subsection}{2}
\subsection{}

\paragraph{Question}
Give two examples in which it might make sense to delay the binding of an
implementation decision, even though sufficient information exists to bind
it early.


\paragraph{Answer}
Local variables in Fortran 77 may be allocated on the stack, rather than in static memory, in
order to minimize memory footprint and to facilitate interoperability with standard tools (e.g.,
debuggers).

Just-in-time compilation allows a system to minimize code size, obtain the most recent implementations
of standard abstractions, and avoid the overhead of compiling functions that
aren’t used.
\subsection{}

\paragraph{Question}
Give three concrete examples drawn from programming languages with
which you are familiar in which a variable is live but not in scope.


\paragraph{Answer}
\begin{enumerate}
  \item C: global variable \verb|x| is still live but not in the scope if a
  local variable \verb|x| is declared in the scope and here appears so called
  hole in the scope.
  \item C: when we declared static variable inside function, it is not in scope
  outside function, but is still alive.
  \item Java: private member variables of an object of Java class are live but
not in scope when execution is not inside a member function of Java.
\end{enumerate}
\subsection{}

\paragraph{Question}
Consider the following pseudocode, assuming nested subroutines and static
scope.

\begin{description}
\item[(a)] What does this program print?
\item[(b)] Show the frames on the stack when \textit{A} has just been called.
For each frame, show the static and dynamic links.
\item[(c)] Explain how \textit{A} finds \textit{g}.
\end{description}

\paragraph{Answer}
Firstly in C after invocation of \verb|main()| are declared and initialized
variables \verb|a| and \verb|b|. Then procedure \verb|middle()| is declared and
immediately called. Inside procedure \verb|middle()| there is declared variable
\verb|b| again, procedure \verb|inner()| and variable \verb|a| also again in
that order. After that procedure \verb|inner()| is invoked and inside it is
\verb|a| and \verb|b| printed. We know that C uses static scoping and that
declaration must be before use. That means that name \verb|b| in procedure
\verb|inner()| is bound to declaration in procedure \verb|middle()| and name
\verb|a| is bound to declaration in \verb|main()|, because \verb|a| in closest
enclosing block (procedure \verb|middle()|) is declared after declaration od
\verb|inner()|. So firstly 1, 1 are printed.



After that in procedure \verb|middle()| are bound names of \verb|a| and \verb|b|
to declaration in this procedure so 3, 1 is printed.

When printing \verb|a| and \verb|b| in \verb|main()|, also here is declaration
in this procedure is used so 1, 2 is printed


In C\# we can declare nested functions only using keyword \verb|delegate|, but
in this nested function is not possible to declare variable with same name as in
enclosing function.
\setcounter{subsection}{13}
\subsection{}

\paragraph{Question}
Consider the following pseudocode.
\newline
\begin{verbatim}
x : integer -- global

procedure set_x(n : integer)
   x := n
   
procedure print_x
   write integer(x)
   
procedure first
   set_x(1)
   print_x
   
procedure second
   x : integer
   set_x(2)
   print_x

set_x(0)
first()
print_x
second()
print_x
\end{verbatim}
What does this program print if the language uses static scoping? What does
it print with dynamic scoping? Why?

\paragraph{Answer}
In static scoping is variable bound to the closest enclosing block, so in
procedure \verb|print_x| is variable \verb|x| always bound to global
declaration, so the output will be:\\
\\
1\\
1\\
2\\
2\\

In dynamic scoping choose the most recent active binding for \verb|x| at
run-time. When calling procedure \verb|second()| and then call inside
\verb|print_x|, variable \verb|x| is bound to
declaration in procedure \verb|second()| and then printed as 2, but when calling
\verb|print_x| outside scope of \verb|second()|, \verb|x| is bound to global
declaration and \verb|print_x| prints 1. So the output will be:\\
\\
1\\
1\\
2\\
1\\
\end{document}
