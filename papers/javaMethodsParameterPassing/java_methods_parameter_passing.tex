\title{Swapping two objects with void function in Java} \author{
        Martin Kozeny\\
        CSCI 4501: Programming Language Structure\\
        Spring 2011
        University of New Orleans
}
\date{\today}




\documentclass[5pt]{article}
\usepackage{graphicx}
\usepackage{amssymb}
\usepackage{amsmath}
\usepackage{qtree}
\usepackage{multicol}
%\usepackage{chemarrow}
\usepackage[utf8]{inputenc}
\usepackage{listings}
  \usepackage{courier}
 
 \lstset{language=Java,
         basicstyle=\footnotesize\ttfamily, % Standardschrift
         numbers=left,               % Ort der Zeilennummern
         numberstyle=\tiny,          % Stil der Zeilennummern
         %stepnumber=2,               % Abstand zwischen den Zeilennummern
         numbersep=5pt,              % Abstand der Nummern zum Text
         tabsize=2,                  % Groesse von Tabs
         extendedchars=true,         %
         breaklines=true,            % Zeilen werden Umgebrochen
         keywordstyle=\color{blue},
         %       frame=b,         
         keywordstyle=[1]\textbf,    % Stil der Keywords
         keywordstyle=[2]\color{blue},    %
 %        keywordstyle=[3]\textbf,    %
 %        keywordstyle=[4]\textbf,   \sqrt{\sqrt{}} %
         %stringstyle=\color{white}\ttfamily, % Farbe der String
         showspaces=false,           % Leerzeichen anzeigen ?
         showtabs=false,             % Tabs anzeigen ?
         xleftmargin=17pt,
         framexleftmargin=17pt,
         framexrightmargin=5pt,
         framexbottommargin=4pt,
         %backgroundcolor=\color{lightgray},
         showstringspaces=false      % Leerzeichen in Strings anzeigen ?        
 }
 \lstloadlanguages{% Check Dokumentation for further languages ...
         %[Visual]Basic
         %Pascal
         %C
         %C++
         %XML
         %HTML
         %Lisp
         Java
 }


\setlength{\hoffset}{-2.3cm} 
\setlength{\voffset}{-3cm}
\setlength{\textheight}{24.0cm} 
\setlength{\textwidth}{16cm}


\begin{document}


\maketitle

\section{Theory}
Before I write a method in Java, which swap two objects, it is necessary
to introduce some theory around that problem. According to \cite{java:jon}, in
Java is everything passed by value and objects are never passed at all. Moreover
the values of variables are always primitives or references, never objects.
\cite{java:jon} claims, that this results in the fact, that we cannot pass the
object! So what the Java do when we call a function with object as a parameter?
The answer is that Java pass \textbf{references to objects}. We can support this
assertion with the second truth, that the values of variables are always
primitives or references, never objects. Passing a reference by value actually
means we can think of references how we think of primitives, to a large extent.
All in all into method is passed a copy of reference to object. That means that,
when this object is modified inside the method, this change will appeared also
on references to this object outside method scope. But when it is to variable
inside the function, which references to object, gived reference to another
object, reference to original object of course expires. We can that fact
demonstrate on the source code shown below.
%This issue is also connected with the fact, that Java Virtual Machine has stack
%based architecture. That means when is some function called, firstly its
%parameters are pushed onto stack. After jumping into method, these parameters
%are poped. That is the same if parameter is primitive data type or object. When
%new local variables inside function are assigned, they are also pushed onto
%stack.
\section{Implementation}
When we study methods $swap_{i}()$, where $i \in {1\ldots5}$, we can
successfully predict, which function will swap two objects and this action
appears also in main scope. If we look at $swap1()$, we say that reference to
the instance of object $b$ is passed to $tmp$ and then to object $b$ is passed
reference to object $a$. After that is to variable $a$ given reference to object
on which refers variable $temp$. After leaving scope of this function, ends
scope of this local variables (also paramaters!) and in outer scope is
everything unchanged. Similarly it is in method $swap3()$ and $swap4()$. In
$swap3()$ object referred with variable $b$ is cloned but this object expired
with end of the scope of method and the same thing happened to objects
created by copy contructors in method $swap4()$. The only way to swap these
objects is to change value of its field, than will be object, which is referred
from outer scope also changed, because it is referred the same object from outer
scope and also from inside context of method. Function $swap2$ and $swap5$ have
the same effect, they only exchange value of object's fields and that's the
only way, how to swap objects in void method.\newpage
\lstinputlisting[language=Java]{src/Main.java}
\newpage
\lstinputlisting[language=Java]{src/TestClass.java}


\textbf{Output of calling}\newline
\begin{sffamily}
\begin{scriptsize} 


Value of object a before swap1: 1\newline
Value of object b before swap1: 2\newline
Value of object a after swap1: 1\newline
Value of object b after swap1: 2\newline
\newline
Value of object a before swap2: 1\newline
Value of object b before swap2: 2\newline
Value of object a after swap2: 2\newline
Value of object b after swap2: 1\newline
\newline
Value of object a before swap3: 2\newline
Value of object b before swap3: 1\newline
Value of object a after swap3: 2\newline
Value of object b after swap3: 1\newline
\newline
Value of object a before swap4: 2\newline
Value of object b before swap4: 1\newline
Value of object a after swap4: 2\newline
Value of object b after swap4: 1\newline
\newline
Value of object a before swap5: 2\newline
Value of object b before swap5: 1\newline
Value of object a after swap5: 1\newline
Value of object b after swap5: 2\newline
\newline

\end{scriptsize}
\end{sffamily}
\begin{thebibliography}{1}

\bibitem{java:jon}
Jon's homepage, Jon Skeet, software engineer at Google,
\newline\verb|http://www.yoda.arachsys.com/java/passing.html#formal|.



\end{thebibliography}



\end{document}
