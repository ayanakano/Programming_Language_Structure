\title{Expression evaluation} \author{
        Martin Kozeny\\
        CSCI 4501: Programming Language Structure\\
        Spring 2011
        University of New Orleans
}
\date{\today}




\documentclass[5pt]{article}
\usepackage{graphicx}
\usepackage{amssymb}
\usepackage{amsmath}
\usepackage{qtree}
\usepackage{multicol}
%\usepackage{chemarrow}
\usepackage[utf8]{inputenc}
\usepackage{listings}
  \usepackage{courier}
 
 \lstset{language=Java,
         basicstyle=\footnotesize\ttfamily, % Standardschrift
         numbers=left,               % Ort der Zeilennummern
         numberstyle=\tiny,          % Stil der Zeilennummern
         %stepnumber=2,               % Abstand zwischen den Zeilennummern
         numbersep=5pt,              % Abstand der Nummern zum Text
         tabsize=2,                  % Groesse von Tabs
         extendedchars=true,         %
         breaklines=true,            % Zeilen werden Umgebrochen
         keywordstyle=\color{blue},
         %       frame=b,         
         keywordstyle=[1]\textbf,    % Stil der Keywords
         keywordstyle=[2]\color{blue},    %
 %        keywordstyle=[3]\textbf,    %
 %        keywordstyle=[4]\textbf,   \sqrt{\sqrt{}} %
         %stringstyle=\color{white}\ttfamily, % Farbe der String
         showspaces=false,           % Leerzeichen anzeigen ?
         showtabs=false,             % Tabs anzeigen ?
         xleftmargin=17pt,
         framexleftmargin=17pt,
         framexrightmargin=5pt,
         framexbottommargin=4pt,
         %backgroundcolor=\color{lightgray},
         showstringspaces=false      % Leerzeichen in Strings anzeigen ?        
 }
 \lstloadlanguages{% Check Dokumentation for further languages ...
         %[Visual]Basic
         %Pascal
         %C
         %C++
         %XML
         %HTML
         %Lisp
         Java
 }


\setlength{\hoffset}{-2.3cm} 
\setlength{\voffset}{-3cm}
\setlength{\textheight}{24.0cm} 
\setlength{\textwidth}{16cm}


\begin{document}


\maketitle

\section{Expression evaluation in Java}
According to chapter 15, part 7 from \cite{java:jls},
is Java's evaluation order from left to right, so the result of calling
\verb|a*f(1)+a| in code below will be $10 * 1*11 + 11 = 121$.\newline
\lstinputlisting[language=Java]{src/Main.java}

\section{Expression evaluation in C}
In C is the situation with evaluating expressions different. According to
\cite{c:lrm}, C uses function first evaluation so the result of calling
\verb|a*f(1)+a| in code below will be $11 * 1*11 + 11 = 132$. In C exists so
called sequence points. According to \cite{wiki:encyclopedia}, sequence points
are checkpoints which it is guaranteed that all side effects of previous evaluations will have been performed, and no
side effects from subsequent evaluations have yet been performed. In our source
code there are two sequence points: function-entry after the evaluation of all
the function's arguments and before execution of anything the function's body
and at function-exit after a returned value has been copied and before the
calling function resumes execution. Function-entry sequence point guarantees,
that input parameter a is evaluated (in our case parameter \verb|z|) and second
function-exit s. p. guarantees incrementation \verb|a| and multiplying by
\verb|x| before leaving function.
\lstinputlisting[language=C]{src/functions.c}


\begin{thebibliography}{1}

\bibitem{java:jls}
Java Language specification, \verb|http://java.sun.com/docs/books/jls/|.

\bibitem{c:lrm}
C Language Reference Manual, April
1999.

\bibitem{wiki:encyclopedia}Wikipedia, The Free Encyclopedia,
\verb|http://en.wikipedia.org/wiki/Sequence_point|

\end{thebibliography}



\end{document}
